
% This LaTeX was auto-generated from MATLAB code with dynamically generated stylesheet that is based on
% one created by Ned Gulley and Matthew Simoneau, September 2003 Copyright 1984-2013 The MathWorks, Inc.
%   location of original file  ...\toolbox\matlab\codetools\private\mxdom2latex.xsl
% Adapted by Han Oostdijk, August 2015
%
% use \lstinputlisting{xx.m} for inclusion of source file xx.m
%

\documentclass{article}
\usepackage[a4paper,margin=1in,landscape]{geometry}
\usepackage[framed,numbered]{matlab-prettifier}
% package matlab-prettifier created by Julien Cretel. Available CTAN (only matlab-prettifier.sty is needed)
\usepackage{graphicx}
\usepackage{epstopdf}
\usepackage{color}
\usepackage{lmodern}
\usepackage{verbatim}
\lstset{style = Matlab-editor}
\usepackage[unicode=true,pdftitle={},
pdfauthor={han@hanoostdijk.nl},
pdfsubject={},
pdfkeywords={},
pdfproducer={},
pdfcreator={},
    bookmarks=false,bookmarksnumbered=true,bookmarksopen=true,bookmarksopenlevel=2,
    breaklinks=false,pdfborder={0 0 1},backref=false,colorlinks=true,hidelinks]
    {hyperref} 

\sloppy
\definecolor{lightgray}{gray}{0.5}
\setlength{\parindent}{0pt}

\begin{document}
\title{}
\author{}
%\maketitle
\lstlistoflistings
%\listoffigures

\def\captionA{} 
\def\captionB{} 
\def\captionC{} 
\def\captionD{} 
\def\captionE{} 
\def\captionF{} 
\def\captionG{} 
\def\captionH{} 
\def\captionI{} 
\def\captionJ{} 
\def\captionK{} 
\def\captionL{} 
\def\captionM{} 
\def\captionN{} 
\def\captionO{} 
\def\captionP{} 

    
    
\section*{exampleB.m : file for publish\_mpl showing extra options}

\begin{par}
As exampleA.m with the only difference that the listings get their own caption (and therefore the section headers are omitted). The listings are presented in a 'lstlistoflistings' by specifying this in the \texttt{publish} options (see example4) in publish\_mapl\_examples.
\end{par} \vspace{1em}

\subsection*{Contents}

\begin{itemize}
\setlength{\itemsep}{-1ex}
   \item Acknowledgement
   \item Square Waves from Sine Waves
   \item Add an Odd Harmonic and Plot It
   \item Note About Gibbs Phenomenon
   \item Listings
\end{itemize}


\subsection*{Acknowledgement}

\begin{par}
This file is adapted from the \texttt{fourier\_demo2.m} file that is included in MATLAB and can be copied in the current directory with
\end{par} \vspace{1em}

\begin{verbatim}copyfile(fullfile(matlabroot,'help','techdoc',...
'matlab_env','examples','fourier_demo2.m'),'.','f')\end{verbatim}
    

\subsection*{Square Waves from Sine Waves}

\begin{par}

% The actual function to publish starts now
% This text block is changed to a latex block to show the caption and reference capabilities
%
% the following statements insert the references to the plots:
The Fourier series expansion for a square-wave is
made up of a sum of odd harmonics, as shown here
by the plots in figure \ref{exampleB_01.eps} on page \pageref{exampleB_01.eps} (1 harmonic),
figure \ref{exampleB_02.eps} on page \pageref{exampleB_02.eps} (5 harmonics) and
figure \ref{exampleB_03.eps} on page \pageref{exampleB_03.eps} (9 harmonics).
%
% the following statements define the captions of the plots:
\global\def\captionA{first harmonic}
\global\def\captionB{sum of first 5 harmonics}
\global\def\captionC{sum of first 9 harmonics}

\end{par} \vspace{1em}
 
\begin{lstlisting}
if exist('avalue','var')
    fprintf('print the value passed to this script: %f\n',avalue)
else
    fprintf('no value passed to this script\n')
end
\end{lstlisting}

        \color{lightgray} 
		\begin{verbatim}print the value passed to this script: 2.000000
\end{verbatim} 
		\color{black}
    

\subsection*{Add an Odd Harmonic and Plot It}

 
\begin{lstlisting}
t   = 0:.1:pi*4;
k   = 1 ;
y   = sin(k*t)/k;
figure(k)
plot(t,y);
title(sprintf('MATLAB caption: plot when k=%.0f',k))
\end{lstlisting}

\begin{figure}[ht]
\centering 
\includegraphics [width=4in]{exampleB_01.eps}
\caption{\captionA} 
\label{exampleB_01.eps} 
\end{figure}
\begin{par}
In each iteration of the for loop add an odd harmonic to y. As \textit{k} increases, the output approximates a square wave with increasing accuracy.
\end{par} \vspace{1em}
\begin{par}
Perform the following mathematical operation at each iteration:
\end{par} \vspace{1em}
\begin{par}
$$ y = y + \frac{\sin kt}{k} $$
\end{par} \vspace{1em}
\begin{par}
Display some of the plots:
\end{par} \vspace{1em}
 
\begin{lstlisting}
for k = 3:2:9
    y = y + sin(k*t)/k;
    if mod(k,4)==1
        figure(k)
        plot(t,y)
        title(sprintf('MATLAB caption: plot when k=%.0f',k))
    end
end
\end{lstlisting}

\begin{figure}[ht]
\centering 
\includegraphics [width=4in]{exampleB_02.eps}
\caption{\captionB} 
\label{exampleB_02.eps} 
\end{figure}

\begin{figure}[ht]
\centering 
\includegraphics [width=4in]{exampleB_03.eps}
\caption{\captionC} 
\label{exampleB_03.eps} 
\end{figure}


\subsection*{Note About Gibbs Phenomenon}

\begin{par}
Even though the approximations are constantly improving, they will never be exact because of the Gibbs phenomenon, or ringing.
\end{par} \vspace{1em}


\subsection*{Listings}

\begin{par}

% assuming m-file in directory one level higher than tex dir (using the standard html subdirectory)
% assuming numbers and framed are not set in \usepackage and they are wanted
% \lstinputlisting[frame=single,numbers=left]{../exampleB.m}
% assuming numbers and framed are set in \usepackage and they are not wanted
% \lstinputlisting[frame=none,numbers=none]{../exampleB.m}
% assuming numbers and framed are set in \usepackage are set and wanted
\lstinputlisting[caption=listing of exampleB script]{../exampleB.m}

\end{par} \vspace{1em}
\begin{par}

\lstinputlisting[caption=listing of publish\_mpl\_examples script]{../publish_mpl_examples.m}

\end{par} \vspace{1em}



\end{document}
    
